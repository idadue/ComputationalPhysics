\documentclass[../main.tex]{subfiles}

\begin{document}
\section{Conclusion}\label{sec:conclusion}

We show that, as expected, $b$ is a very important parameter when it comes to dealing with any disease. Any population which can keep a value $b \approx a$ will significantly reduce the total number of infected. 
We then go on to show that the deadliness of a disease can drastically reduce a population, if the given rate $d_I$ is big enough.

Then we go on to show how a disease can vary with the seasons. There can be periods of times where the disease spreads well, and periods where the spread is decreased.

Finally, we show how the introduction of a vaccine can greatly decrease the number of infected, by bypassing the cycle of $\mathcal{S} \longrightarrow \mathcal{I} \longrightarrow \mathcal{R}$ and instead allowing for susceptibles to jump straight to recovered.


\end{document}
