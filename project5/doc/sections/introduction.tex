\documentclass[../main.tex]{subfiles}

\begin{document}
\section{Introduction}\label{sec:introduction}

Nothing has killed more human beings than infectious diseases. Whenever they happen, they significantly impact global economies and public health. Therefore, it is of great significance to use mathematical models to analyze the transmission and control of infectious diseases. By using these models, we can make predictions about e.g. the total number infected, the duration of an epidemic, or whether or not a certain disease has the capacity to establish itself within the population. These analyzes may help governments decide which public health interventions to set in place to mitigate the disease. 

Compartment models can be used to simplify the mathematical modelling of infectious diseases. In these models, the population is assigned to different compartments, e.g. $S$, $I$ or $R$ (Susceptible, Infectious, or Recovered) which people may progress between. We investigate the SIRS model. Here, the order of labels shows the flow pattern between the compartments; susceptible, infectious, recovered than susceptible again. 

The aim of this work is to develop a Monte Carlo simulation of the spread of an infectious disease for the purpose of investigating how a disease spreads throughout a given population over time. We begin with running the SIRS model with ordinary differential equations (which are deterministic), which we solve using the fourth-order Runge-Kutta method. Once this is done, we switch to Monte Carlo methods to use randomness to solve the problem. This is a more realistic approach but it is much more complicated to analyze. Further, we modify or simple SIRS model to add vital dynamics, seasonal variations or vaccination. We run these modified models using both the fourth-order Runge-Kutta method and Monte Carlo methods. 

The outline of this report is as follows: \Cref{sec:theory} gives a short review of the SIRS model and some modifications to include more details in the model. In \cref{sec:method}, we briefly explain how we have implemented the methods and which parameters we have used. Then we present our results in \cref{sec:results}, before we discuss our findings in \cref{sec:discussion}. Last, some concluding remarks are given in \cref{sec:conclusion}.
\end{document}
