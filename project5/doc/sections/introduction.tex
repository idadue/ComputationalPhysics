\documentclass[../main.tex]{subfiles}

\begin{document}
\section{Introduction}\label{sec:introduction}

Nothing has killed more human beings than infectious diseases. Whenever they happen, they significantly impact global economies and public health. Therefore, it is of great significance to use mathematical models to analyze the transmission and control of infectious diseases. By using these models, we can make predictions about e.g. whether or not a certain disease has the capacity to establish itself within the population

The aim of this work is to develop a Monte Carlo simulation of the spread of an infectious disease for the purpose of investigating how a disease spreads throughout a given population over time. We begin with 

The outline of this report is as follows: \Cref{sec:theory} gives a short review of the SIRS model and some modifications to include more details in the model. In \cref{sec:method}, we briefly explain how we have implemented the methods and which parameters we have used. Then we present our results in \cref{sec:results}, before we discuss our findings in \cref{sec:discussion}. Last, some concluding remarks are given in \cref{sec:conclusion}.
\end{document}
