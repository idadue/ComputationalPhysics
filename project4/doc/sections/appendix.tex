\documentclass[../main.tex]{subfiles}
\begin{document}
\appendix
\renewcommand{\theequation}{A\arabic{equation}}
\setcounter{equation}{0}
\section{Appendix}
\subsection{Source code}
Github repository with codes and figures can be found at \url{https://github.com/idadue/ComputationalPhysics/tree/master/project4}.

\subsection{Possible changes of energy}

\begin{table}[!htb]
\caption{The five different possible changes in energy, $\Delta E$} in a two dimensional lattice, when only interacting with the nearest neighbours. 
\begin{center}
\begin{tabular}{ c c c c c}
\toprule
Initial state & Flipped state & Initial \ensuremath{E} & Flipped \ensuremath{E} & \ensuremath{\Delta E}\\ 
\midrule
$\begin{matrix}& \uparrow & \\ \uparrow & \uparrow & \uparrow \\ & \uparrow & \\ & & & \end{matrix}$         & $\begin{matrix}& \uparrow & \\ \uparrow & \downarrow & \uparrow \\ & \uparrow & \\ & & & \end{matrix}$ & -4J & 4J &     8J                 \\
$\begin{matrix}& \downarrow & \\ \uparrow & \uparrow & \uparrow \\ & \uparrow & \\ & & & \end{matrix}$       & $\begin{matrix}& \downarrow & \\ \uparrow & \downarrow & \uparrow \\ & \uparrow & \\ & & & \end{matrix}$ & -2J &  2J    & 4J                  \\
$\begin{matrix}& \downarrow & \\ \downarrow & \uparrow & \uparrow \\ & \uparrow & \\ & & & \end{matrix}$     & $\begin{matrix}& \downarrow & \\ \downarrow & \downarrow & \uparrow \\ & \uparrow & \\ & & & \end{matrix}$ & 0 & 0   & 0                  \\
$\begin{matrix}& \downarrow & \\ \downarrow & \uparrow & \downarrow \\ & \uparrow & \\ & & & \end{matrix}$   & $\begin{matrix}& \downarrow & \\ \downarrow & \downarrow & \downarrow \\ & \uparrow & \\ & & & \end{matrix}$ & 2J & -2J  & -4J                 \\
$\begin{matrix}& \downarrow & \\ \downarrow & \uparrow & \downarrow \\ & \downarrow & \\ & & & \end{matrix}$ & $\begin{matrix}& \downarrow & \\ \downarrow & \downarrow & \downarrow \\ & \downarrow & \\ & & & \end{matrix}$ & 4J & -4J & -8J                 \\ 

\bottomrule
\end{tabular}
\end{center}
\label{tab:possible-delta-E}
\end{table}

\subsection{Analytical expressions for a simple \ensuremath{2\times2} lattice}\label{sec:analytical-L2}
Consider a $2\times2$ lattice, each lattice point with spin $\pm1$, this model has \ensuremath{s=2^4=16} different configurations as shown in \cref{tab:configurations}. With the size of this model, it is possible to calculate analytically the expectation values with the partition function, as it only contains 16 elements (most of which are zero).

\begin{table}[!htb]
\caption{All configurations of a $2\times2$ Ising model. The table displays the degeneracy, energy and magnetization where \ensuremath{N_{\uparrow}} is the number of spins up.} 
\begin{center}
\begin{tabular}{ c c c c }
\toprule
\ensuremath{N_{\uparrow}} & \ensuremath{\Omega(E_i)} & \ensuremath{E_i} & \ensuremath{M_i}\\ 
\midrule
4 & 1 & \SI{-8}{\joule} & 4 \\  
3 & 4 & 0 & 2 \\
2 & 4 & 0 & 0 \\
2 & 2 & \SI{8}{\joule} & 0 \\
1 & 4 & 0 & -2 \\
0 & 1 & \SI{-8}{\joule} & -4 \\
\bottomrule
\end{tabular}
\end{center}
\label{tab:configurations}
\end{table}

The partition function for the $2\times2$ lattice is given by 
\begin{align*}
    Z=\sum_i e^{-E_i\beta}=12+2e^{8\beta}+2e^{-8\beta}=4(3+\cosh8\beta),
\end{align*} where \ensuremath{\beta=1/k_BT}.

The expected energy is then 

\begin{align*}
    \braket{E}=-\frac{\partial}{\partial\beta}\ln Z = - \frac{\partial}{\partial\beta}\ln(4\cosh(\beta8J)+12)=-8J\frac{\sinh(8\beta J)}{\cosh(8\beta J)+3}, 
\end{align*} and the variance of the mean energy is 

\begin{align*}
    \sigma_E^2&=\braket{E^2}-\braket{E}^2 \\
    &=\frac{1}{2e^{-\beta8J}+2e^{\beta8J}+12}(2\cdot(-8J)^2\cdot e^{\beta8J}+2\cdot(8J)^2\cdot e^{-\beta8J} ) - \left(-8j\frac{\sinh(8\beta J)}{\cosh(8\beta J)+3}\right)^2 \\
    &= 64J^2\left(\frac{\cosh(8\beta J)}{\cosh(8\beta J)+3}-\left(\frac{\sinh(8\beta J)}{\cosh(8\beta J)+3}\right)^2\right).
\end{align*}

We can now simply insert the variance of the mean energy to get the expression for the specific heat capacity
\begin{align*}
    C_V&=\frac{\sigma_E^2}{k_BT^2} \\
    &=\frac{1}{k_BT^2}64J^2\left(\frac{\cosh(8\beta J)}{\cosh(8\beta J)+3} - \left(\frac{\sinh(8\beta J)}{\cosh(8\beta J)+3}\right)^2\right).
\end{align*}

The mean magnetisation is found by adding all possible states and dividing by the partition function

\begin{align*}
    \braket{M}=\frac{1}{Z}\sum_iM_ie^{-\beta E_i}=\frac{1}{Z}(-4e^{8\beta J}-8e^0+8e^0+4e^{8\beta J}) = 0.
\end{align*}

The expected absolute magnetisation is non-zero and given by
\begin{align*}
    \braket{|M|}=\frac{1}{Z}\sum_i|M_i|e^{-\beta E_i}=\frac{1}{Z}(4e^{8\beta J}+8e^0+8e^0+4e^{8\beta J})=\frac{4+2e^{8\beta J}}{\cosh(8\beta J) + 3}.
\end{align*}

We calculate the variance of the mean magnetization, needed for the calculation of the magnetic susceptibility
\begin{align*}
    \sigma_M^2&=\braket{M^2}-\braket{M}^2 \\
    &=\frac{32}{Z}(e^{8\beta J}+1)-0 \\
    &=\frac{8(e^{8\beta J}+1)}{\cosh(8\beta J)+3}.
\end{align*}

The magnetic susceptibility is then
\begin{align*}
    \chi&=\frac{1}{k_BT}\sigma_M^2 \\
    &= \frac{1}{k_BT}\frac{8(e^{8\beta J}+1)}{\cosh(8\beta J)+3}. 
\end{align*}


\end{document}
